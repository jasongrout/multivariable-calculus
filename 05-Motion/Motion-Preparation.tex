\section{Preparation}

\subsection{Lesson Plans}

This chapter covers the following ideas. When you create your lesson plan, it should contain examples which illustrate these key ideas. Before you take the quiz on this unit, meet with another student out of class and teach each other from the examples on your lesson plan. 

% A list of objectives for the chapter
%\begin{enumerate}
%\item ...
%\end{enumerate}


\begin{enumerate}
\item Describe projectile motion.  Develop formulas which are valid if
  we neglect air resistance and consider only acceleration due to
  gravity.  Use your model to develop formulas for the range, maximum
  height, and flight time.
\item Develop the $TNB$ frame for describing motion. Add to your model
  the concepts of curvature, osculating circle, torsion, and the
  tangential and normal components of acceleration. Be able to prove
  the relationships that you develop in the $TNB$ frame.
\end{enumerate}


%%% Local Variables: 
%%% mode: latex
%%% TeX-master: "../multivariable-calculus"
%%% End: 
%$


%\subsection{Preparation Problems}

%Here are the preparation problems for this unit.

\subsection{Homework}

In the following list, the ``basic practice'' problems should be quick
problems to help you master the ideas.  The ``good problems'' will
require a little more work.  The theory and application problems are
ones that will challenge you more; make sure you do the problems from
this area to fully master the material.  

\begin{center}
  \begin{tabular}{p{1in}cp{1.3in}p{1.3in}p{1.3in}}\toprule
    Topic & Section & Basic Practice & Good problems & Theory/Application \\\midrule
    Velocity \& Acceleration & 12.3 & 1--16, 19--23 & 17--18, 25--30, 41--42, 57--59 & 24, 31--40, 43--57 \\
    Tangent \& Normal vectors & 12.4 & 1--10, 23--44, 49--56, 65--70 & 11--22, 45--48, 57--58 & 60--62, 75--90 \\
    Arclength \& Curvature & 12.5 & 1--6, 9--14, 21--30, 41--46 & 7--8, 15--16, 31--40, 49--52, 55--64, 88--93 & 17--20, 47--48, 65--66, 72, 75--80, 84--86, 94--100
    \\\bottomrule
  \end{tabular}
\end{center}

Don't do too many of the curvature and torsion problems by hand, as
they can be very time consuming.  Practice a few so that you can make
sure you understand the computations, but then spend the most of your
time reviewing the theory and proving the relationships that exist
among the vectors.

\subsection{Webcasts}

Ben Woodruff has posted a series of webcasts covering topics in this
chapter: \url{http://www.youtube.com/user/bmwoodruff#g/c/30EE81142B1ED1F0}.

%%% Local Variables: 
%%% mode: latex
%%% TeX-master: "../multivariable-calculus"
%%% End: 
