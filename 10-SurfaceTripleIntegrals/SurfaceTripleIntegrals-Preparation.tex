\section{Preparation}

\subsection{Lesson Plans}

This chapter covers the following ideas. When you create your lesson plan, it should contain examples which illustrate these key ideas. Before you take the quiz on this unit, meet with another student out of class and teach each other from the examples on your lesson plan. 

% A list of objectives for the chapter
%\begin{enumerate}
%\item ...
%\end{enumerate}


\begin{enumerate}
\item Explain how to setup surface integrals, as well as how to
compute surface area.
\item Explain how to set up and compute triple
  integrals, as well as how to interchange the bounds of
  integration. Use these ideas to find volume.
\item For surfaces and solid regions, find area,
  mass, centroids, center of mass, moments of interia, and radii of
  gyration.
\item Explain how to change coordinate systems in integration, in
  particular to polar, cylindrical, and spherical coordinates. Explain
  what the Jacobian of a transformation is and how to use it.
\end{enumerate}

%%% Local Variables: 
%%% mode: latex
%%% TeX-master: "../multivariable-calculus"
%%% End: 
%$


%\subsection{Preparation Problems}

%Here are the preparation problems for this unit.

\subsection{Homework}

In the following list, the ``basic practice'' problems should be quick
problems to help you master the ideas.  The ``good problems'' will
require a little more work.  The theory and application problems are
ones that will challenge you more; make sure you do the problems from
this area to fully master the material.  

{\centering
\begin{tabular}{lcllll}\toprule
Topic &Sec &Basic Practice &Good Problems &Thy/App \\\midrule

Surface area & 14.5 & 1--24 & 27--34 & 37--41\\\hline
Parametric surfaces & 15.5 & 1--4, 35--42 & 13--26\\\hline
Surface integrals & 15.6 & 1--22 & 37--40\\\bottomrule
Triple integrals & 14.6 & 1--38, 63--66 & 43--58 & 39-40, 61--62, 67--70\\\hline
Spherical and Cylindrical & 14.7 & 1--22 & 23--38 & 43--46\\\hline
Jacobians & 14.8 & 27--30&&\\\hline
\end{tabular}

} 
\bigskip
It is crucial that you do not attempt to solve every integral.  For
the most part, you are learning to set up integrals in high
dimensions. You should do many more than 5 problems a day if you want
to become proficient at multiple integration.  \textbf{I would suggest
  that you do at least 15 or more problems a day in the assigned
  sections}, where you spend time setting up integrals and not solving
them.


%%% Local Variables: 
%%% mode: latex
%%% TeX-master: "../multivariable-calculus"
%%% End: 
